\documentclass[12pt,a4paper]{article}
\usepackage[utf8]{inputenc}
\usepackage[T2A]{fontenc}
\usepackage[russian]{babel}
\usepackage{amsmath,amssymb}
\usepackage{graphicx}
\usepackage{geometry}
\usepackage{float}
\geometry{margin=2cm}
\usepackage{indentfirst}
\usepackage{caption}

\begin{document}

\begin{titlepage}
\begin{center}

\textbf{\Large МОДЕЛИРОВАНИЕ ДВИЖЕНИЯ РЕЛЯТИВИСТСКОЙ ЗАРЯЖЕННОЙ ЧАСТИЦЫ\\
В СКРЕЩЁННЫХ ПОЛЯХ}\\[6em]

Сметанин Григорий\\[1em]

\end{center}
\end{titlepage}

\tableofcontents
\newpage

\section*{Введение}
\addcontentsline{toc}{section}{Введение}

Цель работы — смоделировать движение заряженной частицы в скрещённых электрическом и магнитном полях с учётом релятивистской динамики, рассмотреть три режима в зависимости от соотношения между напряжённостью электрического и магнитного полей и сравнить релятивистское движение с классическим нерелятивистским дрейфом.

В работе численно исследуются три случая:
\begin{enumerate}
    \item Электрическое поле меньше магнитного: $E < cB$;
    \item Электрическое и магнитное поля равны по модулю: $E = cB$;
    \item Электрическое поле больше магнитного: $E > cB$.
\end{enumerate}

Для численного решения используется \textbf{метод Рунге–Кутты четвёртого порядка точности (RK4)}. Этот метод обеспечивает точное интегрирование систем обыкновенных дифференциальных уравнений.

\medskip

Метод Рунге–Кутты основан на поэтапном вычислении промежуточных значений производной функции и усреднении их с определёнными весами. Для уравнения
\[
\frac{dy}{dt} = f(y, t), \quad y(t_0)=y_0,
\]
шаг интегрирования $\Delta t$ выполняется по формулам:
\begin{equation}
\begin{aligned}
k_1 &= f(y_n, t_n),\\
k_2 &= f\left(y_n + \frac{\Delta t}{2}k_1,\, t_n + \frac{\Delta t}{2}\right),\\
k_3 &= f\left(y_n + \frac{\Delta t}{2}k_2,\, t_n + \frac{\Delta t}{2}\right),\\
k_4 &= f\left(y_n + \Delta t\,k_3,\, t_n + \Delta t\right),\\[0.5em]
y_{n+1} &= y_n + \frac{\Delta t}{6}\left(k_1 + 2k_2 + 2k_3 + k_4\right).
\end{aligned}
\label{eq:rk4}
\end{equation}

Усреднение в формуле (\ref{eq:rk4}) уменьшает ошибку шага до $O(\Delta t^5)$, а полная погрешность решения имеет порядок $O(\Delta t^4)$.

В задаче о движении заряженной частицы этот метод применяется к системе уравнений:
\[
\frac{d}{dt}
\begin{pmatrix}
\mathbf{r}\\[0.3em]
\mathbf{v}
\end{pmatrix}
=
\begin{pmatrix}
\mathbf{v}\\[0.3em]
\dfrac{q}{m\gamma}\left(\mathbf{E} + \mathbf{v}\times\mathbf{B} - \dfrac{(\mathbf{v}\cdot\mathbf{E})\mathbf{v}}{c^2}\right)
\end{pmatrix}.
\]


\newpage
\section{Теоретическая часть}

\subsection{Основные уравнения движения}

Рассмотрим заряженную частицу с зарядом $q$ и массой $m$, движущуюся со скоростью $\mathbf{v}$ в электрическом и магнитном полях $\mathbf{E}$ и $\mathbf{B}$. Общий закон Лоренца имеет вид:
\begin{equation}
    \frac{d\mathbf{p}}{dt} = q \left( \mathbf{E} + \mathbf{v} \times \mathbf{B} \right),
    \label{eq:lorentz}
\end{equation}
где $\mathbf{p}$ — импульс частицы. В релятивистской форме
\begin{equation}
    \mathbf{p} = \gamma m \mathbf{v}, \qquad
    \gamma = \frac{1}{\sqrt{1 - \frac{v^2}{c^2}}}.
\end{equation}

Ускорение $\frac{d\mathbf{v}}{dt}$ выражается как:
\begin{equation}
    \frac{d\mathbf{v}}{dt} = \frac{q}{m\gamma}\left(
    \mathbf{E} + \mathbf{v} \times \mathbf{B} - \frac{\mathbf{v}(\mathbf{v} \cdot \mathbf{E})}{c^2}
    \right).
    \label{eq:accel}
\end{equation}


Для численного моделирования уравнения (\ref{eq:lorentz}) и (\ref{eq:accel}) представляются в виде системы шести обыкновенных дифференциальных уравнений:
\begin{equation}
    \begin{cases}
        \dfrac{d\mathbf{r}}{dt} = \mathbf{v},\\[0.5em]
        \dfrac{d\mathbf{v}}{dt} = \dfrac{q}{m\gamma}\left(\mathbf{E} + \mathbf{v}\times\mathbf{B} - \dfrac{(\mathbf{v}\cdot\mathbf{E})\mathbf{v}}{c^2}\right).
    \end{cases}
\end{equation}

\subsection{Характер движения в скрещённых полях}

Наши электрическое и магнитное поля взаимно перпендикулярны:
\[
\mathbf{E} = (E, 0, 0), \qquad \mathbf{B} = (0, 0, B).
\]

В зависимости от отношения между $E$ и $cB$ рассмотрим три режима:

\begin{itemize}
    \item \textbf{Магнитодоминированный режим} ($E < cB$).
    \item \textbf{Критический режим} ($E = cB$). 
    \item \textbf{Электродоминированный режим} ($E > cB$).
\end{itemize}

\subsection{Преобразование полей и физический смысл трёх режимов}

Электрическое и магнитное поля зависят от выбора инерциальной системы отсчёта. При переходе в систему, движущуюся со скоростью $\mathbf{u}$ относительно лабораторной, они преобразуются по формулам Лоренца:
\[
\mathbf{E}' = \gamma_u(\mathbf{E} + \mathbf{u}\times\mathbf{B}) 
- \frac{\gamma_u^2}{\gamma_u+1}\frac{\mathbf{u}(\mathbf{u}\cdot\mathbf{E})}{c^2},
\qquad
\gamma_u = \frac{1}{\sqrt{1 - \frac{u^2}{c^2}}}.
\]

Для полей $\mathbf{E}\perp\mathbf{B}$ можно выбрать скорость
\[
\mathbf{u} = \frac{\mathbf{E}\times\mathbf{B}}{B^2},
\]
при которой $\mathbf{E}'=0$. Такое преобразование возможно только при $E<cB$, так как тогда $u=E/B<c$. В этой системе наблюдается лишь магнитное поле, и частица вращается вокруг линий $\mathbf{B}$, что соответствует устойчивому дрейфу.

При $E=cB$ требуемая скорость компенсации равна $c$, а при $E>cB$ она превышает скорость света, что делает устранение электрического поля невозможным. Следовательно, при $E\ge cB$ движение становится нестационарным и сопровождается непрерывным ростом энергии частицы.



\section{Численное моделирование}

Для исследования движения была выполнена численная интеграция системы (\ref{eq:accel}). Расчёты выполнены в языке программирования Python при помощи библиотек \texttt{NumPy} и \texttt{Matplotlib}. В качестве тестовой частицы рассматривался электрон:
\[
q = -1.602\times10^{-19}\ \text{Кл}, \qquad
m = 9.109\times10^{-31}\ \text{кг}.
\]

а начальные условия:
\[
\mathbf{r}_0 = (0,0,0), \qquad \mathbf{v}_0 = (10^6, 0, 0)\ \text{м/с}.
\]

На каждом шаге времени вычислялись координаты $\mathbf{r}(t)$, скорость $\mathbf{v}(t)$ и фактор Лоренца
\[
\gamma(t) = \frac{1}{\sqrt{1-\frac{v^2(t)}{c^2}}}.
\]
Графики траекторий строились в проекции $xy$, а также строилась зависимость $\gamma(t)$, отражающая изменение энергии частицы.

\subsection{Результаты моделирования}

\begin{figure}[H]
\centering
\includegraphics[width=0.75\textwidth]{1.png}
\caption{Траектория частицы при $E = 10^5, B = 0.5$.}
\end{figure}

\begin{figure}[H]
\centering
\includegraphics[width=0.75\textwidth]{11.png}
\caption{Зависимость фактора Лоренца $\gamma(t)$ при $E = 10^5, B = 0.5$.}
\end{figure}

\paragraph{Случай $E < cB$.}
\newpage

\begin{figure}[H]
\centering
\includegraphics[width=0.75\textwidth]{2.png}
\caption{Траектория частицы при $E = cB, B = 0.05$.}
\end{figure}

\begin{figure}[H]
\centering
\includegraphics[width=0.75\textwidth]{22.png}
\caption{Изменение фактора Лоренца $\gamma(t)$ при $E = cB, B = 0.05$.}
\end{figure}

\paragraph{Случай $E = cB$.}
При равенстве модулей полей частица испытывает ускорение вдоль направления $\mathbf{E}\times\mathbf{B}$. На графике траектории видно постепенное выпрямление движения и рост $\gamma(t)$.

\begin{figure}[H]
\centering
\includegraphics[width=0.75\textwidth]{3.png}
\caption{Траектория частицы при $E = 1.2cB, B = 0.05$.}
\end{figure}

\begin{figure}[H]
\centering
\includegraphics[width=0.75\textwidth]{33.png}
\caption{Изменение фактора Лоренца $\gamma(t)$ при $E = 1.2cB, B = 0.05$.}
\end{figure}

\paragraph{Случай $E > cB$.}
Электрическое поле доминирует, и частица разгоняется почти прямо вдоль $\mathbf{E}$, лишь слегка отклоняясь под действием $\mathbf{B}$. В этом режиме $\gamma(t)$ растёт быстрее.


\newpage
\section{Сравнение с нерелятивистским дрейфом}

Для контроля корректности модели был рассмотрен предельный нерелятивистский случай ($v \ll c$). При тех же значениях $E$ и $B$ траектория частицы имеет форму спирали с постоянным дрейфом, который хорошо совпадает с аналитическим выражением
\[
\mathbf{v}_d = \frac{\mathbf{E}\times\mathbf{B}}{B^2}.
\]
Таким образом, релятивистская модель корректно воспроизводит классическую динамику при малых скоростях и даёт ожидаемое отклонение при приближении скорости к световой.


\end{document}